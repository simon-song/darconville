\chapter*{Explicitur}
\addcontentsline{toc}{chapter}{Explicitur}%

THIS IS A STORY of murder, which, as an act, is as apt 
\mynote{apt}{(adj.): likely}
to characterize deliverance 
\mynote{deliverance}{(n.): setting free}
as it is to corroborate 
\mynote{corroborate}{(v.):  OED 3. With non-material object: To strengthen 
(a quality, faculty, power, etc.); to confirm (a person) in a quality or 
attribute. arch.}
death. There are certain elemental emotions
that touch upon powers other and larger than our own discrete wishes might
allow --- for every consciousness is continuous 
\footnote{cf. \textit{discrete wishes} above}
with a wider self open to the hidden
processes and unseen regions created in the soul by the very nature of an
opposite effort --- and while, taken together, each may prove the other simply by
contrast, considered separately neither may admit of various shades in the law
of whichever whole it finds reigning at the time. That which produces effects
within one reality creates another reality itself. I am thinking, specifically,
of love and hate.

  We cannot distinguish, perhaps, natural from supernatural effects, nor among
either know which are favors of God and which are counterfeit operations of the
Devil. Who, furthermore, can speak of the incubations
\mynote{incubation}{(n.): hatching, brooding. \textit{incubate} as a verb means 
to sit upon (eggs) in order to hatch them, thus figuratively means to brood.
}
of motives? And of love
and hate? Are they not too often, in spite of the comparative chaos within us,
generally taken to be little more than a set of titles obtained by the mere
mechanical manipulation of antonyms? I have no aspiration
\mynote{aspiration}{(n.): the action of aspiring; steadfast desire or longing 
for something above one.
}
here to reclaim
mystery and paradox from whatever territory they might inhabit, for there is,
indeed, often a killing in a kiss, a mercy in the slap that heats your face.
\footnote{Note the musicality here.}

  There is, nevertheless, a particular poverty in those alloplasts 
%\mynote{alloplast}{(n.): 1. an inert foreign body used for transplantation into 
%tissues. 2. a graft of an inert metal or plastic material.
%}
\mynote{alloplast}{(n.): \textdbend from alloplastic, OED. 1. Orig. Psychoanal. 
(Concerned with) altering or manipulating the external world, as opposed to 
altering the person, e.g. through psychological or evolutionary adaptation.}
who, addressing tragedy, seek to subdistinguish motives beyond those we have best,
because nearest, at hand, and so it is with love and hate --- emotions upon
whose necks, whether wrung
\mynote{wrung}{ppl. of \textit{wring}, which means to press, squeeze, or twist (a 
moist substance, juicy fruit, etc.), esp. so as to drain or make dry.}
or wreathed,
\mynote{wreathed}{ppl. of \textit{wreathe}, which means to twist or coil (something).}
may be found the oldest fingerprints of man. A simple truth intrudes: the basic 
instincts of every man to every man are known.
But who knows when or where or how? For the answers to such questions, summon
Augurello, 
\footnote{Giovanni Aurelio Augurello (Joannes Aurelius Augurellus) (1441-1524)
was an Italian humanist scholar, poet and alchemist.}
your personal jurisconsult
\mynote{jurisconsult}{(n.): one learned in law, esp. in civil or international 
law; a jurist; a master of jurisprudence.}
and theological wiseacre,
\mynote{wiseacre}{(n.): one who thinks himself, or wishes to be thought, wise; a
pretender to wisdom; a foolish person with an air or affectation of wisdom.}
to teach you about primal reality and then to dispel those complexities and cabals 
\mynote{cabal}{(n.): OED 2.a., any tradition or special private interpretation.}
you crouch behind in this sad, psychiatric century you call your own. It is the
anti-labyrinths of the world that scare. Here is a story for you. Your chair.

\vspace{0.4cm}
\rightline{A.L.T.}

