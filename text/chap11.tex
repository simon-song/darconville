%\chapter{Chantepleure}

\chapter[Chantepleure] 
{ 
\Large{Chantepleure}
\footnote{
OED 1. Name of a French poem of the 13th c. addressed to those who sing (chanter) in this world and shall weep (pleurer) in the next (Godef.): hence used of a mixture or alternation of joy and sorrow. 
}
}

\begin{quotation}
  Repetition is the essence of conjuration.
\mynote{conjuration}{(n.): OED 3.a. The effecting of something supernatural by the invocation of a sacred name or by the use of some spell; orig. the compelling of spirits or demons, by such means, to appear and do one's bidding.}
\sourceatright{--- St. NEOT OF AXHOLME}
\end{quotation} 
\vspace{0.2cm}
 
  "ISABEL, Isabel, Isabel, Isabel, Isabel, Isabel, Isabel, Isabel, Isabel,
Isabel, Isabel, Isabel, Isabel." Darconville repeated the name into the
unintelligibility we call sleep—and dreamt of desperadoes.
\mynote{desperado}{(n.): OED 2. A desperate or reckless man; one ready for any deed of
  lawlessness or violence;} 
 



