\chapter{Darconville}

\begin{quotation}
I thought I heard the rustle 
\footnote{A quick succession or confusion of small sounds, like those made by
shaking leaves or straw, by rubbing silk, or the like; a rustling.}
of a dress, but I don’t --- I don’t see anyone. No, I imagined it. \\
\sourceatright{--- \textit{Peter Schlemihl; 
or, The Man Who Sold His Shadow}
\footnote{Peter Schlemihl is the title character of an 1814 novella, 
\textit{Peter Schlemihls wundersame Geschichte (Peter Schlemihl's Miraculous 
Story)}, written in German by exiled French aristocrat Adelbert von Chamisso.
}
}
\end{quotation} 
\vspace{0.2cm}

SEPTEMBER: it was the most beautiful of words, he’d always felt, evoking
orange-flowers, swallows, and regret. The shutters 
\footnote{Protective panels, usually wooden, placed over windows to block out
the light.}
were open. Darconville stared
out into a small empty street, touched with autumnal fog, that looked like the
lugubrious 
\footnote{mournful; indicating sorrow, often ridiculously or feignedly; doleful;
woful; pitiable; as, "a whining tone and a lugubrious look".
From L. \textit{lugubris}, from L. \textit{lugere} to mourn; cf. Gr. λυγρός sad, 
Skr. \textit{ruj} to break.
}
frontispiece to a book as yet to be read. His obligate 
\footnote{\textdbend absolutely indispensable; essential. Several dictionaries
have the meaning "restricted to a particular condition of life".}
room, its walls several shades of distemper, 
\footnote{A method of painting, in which the colours are mixed with some
glutinous substance soluble in water, as yolk of egg mixed with water, etc.,
executed usually upon a ground of chalk or plaster mixed with gum
(distemper-ground): mostly used in scene-painting, and in the internal
decoration of walls. Chiefly in such phrases as ‘painting’ or ‘to paint in
distemper’ (Italian \textit{pingere a tempera}). 
}
was spare as the skite 
\footnote{\textdbend Probably it means some kind of place or cell. OED does not 
give any meaning that fits here.}
of a recluse
\footnote{A person who lives in seclusion from intercourse with the world, as a
hermit or monk; specifically, one of a class of secluded devotees who live in
single cells, usually attached to monasteries.
}
--- a postered 
\footnote{\textdbend decorated with posters?}
bed, several chairs, and an old deal desk he’d just left, confident in
the action of moderating powers, to ease his mind of some congested thoughts. He
looked at his watch which he kept hung on a nail. The afternoon was to have been
spent, as the morning had, writing, but something else was on his mind.

  There was an unfinished manuscript, tentatively called \textit{Rumpopulorum}, 
spread out there, a curious, if speculative, examination of the world of angels,
archistrateges, 
\footnote{\textdbend \textit{archi}+\textit{stratege}, chief-general.
\textit{stratege=strategus} a commander-in-chief or chief magistrate at Athens 
and in the Achæan league (also in Harrington's imaginary commonwealth),
from Gr. στρατηγ-ός. Cf. F. stratège (also stratègue).
%closest word is archistrategos: “chief-general” in Byzantine Greek.
}
and the archonic 
\footnote{\textdbend From Gr. ἄρχων ruler, magistrate, pr. pple. of ἄρχ-ειν 
to rule.}
wardens 
\footnote{A keeper, a guardian, a watchman.}
of heaven in relation --- he appropriated
\footnote{v. to take to one's self in exclusion of others; to claim or use as by
an exclusive right.}
without question the right to know both --- to mortal man. The body of material,
growing over the last few months, was formidable, its sheets pied 
\footnote{\textit{pi} as a verb means to put into a mixed and disordered 
condition, as type; to mix and disarrange the type of.}
with inky corrections and smudged 
\footnote{To soil, stain, blacken, smirch; to mark with dirty stains or smears.}
with the additions that overheated his prose and yet
brought it all to test.

  The human skull, his pencils in its noseholes, that had been ritually placed
on that desk a week previous—his first days in the South— seemed appropriate to
his life, a reminder, mysteriously elate, 
\footnote{fig. Of condition, and of persons with regard to their condition:
Exalted, lofty. Of feelings, etc.: Lofty, proud.}
of what actually wasn’t, something
there but not, a memory of man without one, for not only had he more or less
withdrawn from the world, long a characteristic of the d’Arconvilles, but the
caricatures of mortal vanity 
\footnote{That which is vain, futile, or worthless; that which is of no value or
profit.} 
were as necessary to his point of view as the
unction 
\footnote{That quality in language, address, or the like, which excites emotion;
especially, strong devotion; religious fervor and tenderness; sometimes, a
simulated, factitious, or unnatural fervor.
}
of religious conventionality was featureless.

  Darconville’s cat leaped onto the windowsill and peered up, as if collating
\footnote{assemble in proper sequence}
the thoughts of his master: where were they? How had they come to be here? What
reason, in fact, had they to be in this strange place? The young man, however,
continued leaning by the window and reviewing what he saw. But there was another
view, for behind it, or perhaps beyond somewhere, in vague, half-blind
remembrances of wherever he’d been --- sources of endless pleasure to him 
--- he dwelled awhile to find himself, looking back in time, surprised at the 
absence in it of
any figure but his own. He felt no particular responsibility to memory but
accepted his dreams, to which, living altogether as a twin self in the depths of
him, he could speak in inviolable secrecy.

  It had long seemed clear, commandmental: to seek out a relatively distant and
unembellished 
\footnote{lacking emellishment or ornamentation}
part of the world where, in the solitude he arranged for
himself—rather like the pilgrim who lives on lentils, 
\footnote{a type of flat, round seed that is related to the pea and is eaten as
a vegetable}
pulses, 
\footnote{the edible seeds of various crops (as peas, beans, or lentils) of the
legume family.}
and the tested modes of self-denial --- one might apply himself to those deeper 
mysteries where nameless somethings in their causes slept. He sought the obol 
\footnote{A silver (in later times bronze) coin of ancient Greece, of the value
of 1/6 of a drachma. }
of Pasetes, the
mallet 
\footnote{a hammer with a large, usually wooden head, used especially for
hitting a chisel.}
of Daikoku, 
\footnote{大黑天, Japanese god of great darkness or balckness, or the god of 
five cereals, one of the Seven Lucky Gods.}
the lamp of Aladdin. There were difficulties, often, in the
way of carrying out his plans. But he overbore them and, hoping to fall prey to
neither fascination nor fatigue, sought only to stem distraction, to learn the
secrets beyond the world he felt belonged to him, and to write. It was the
Beatitude 
\footnote{Felicity of the highest kind; consummate bliss.}
of Destitution.
\footnote{The state of being deprived of anything; the state or condition of
being destitute, needy, or without resources; deficiency; lack; extreme poverty;
utter want.}

  Alaric Darconville --- insurrect, 
\footnote{\textdbend OED only gives meanings as a verb: (1) to arise, (2) to
rise in insurrection or revolt. Here it apparently is an adjective.}
courteous, liturgical
\footnote{Pertaining to, of or the nature of, a liturgy; of or pertaining to
public prayer and worship.}
--- was twenty-nine years old.
He had the pointed medieval face of a pageboy, 
\footnote{a male page, especially in a hotel or attending a bride at a wedding.}
which showed less of mature
steadiness than innocent deliberation, an expensive coloring backlit with a kind
of intangible grace, and his eyes, of a strange tragic beauty, dark and filled
with studying what’s represented by what is, could light up like a monk at
jubilee 
\footnote{In Judaism and Christianity, the concept of the Jubilee is a special
year of remission of sins and universal pardon. In the Book of Leviticus, a
Jubilee year (Hebrew: יובל‎‎ yūḇāl) is mentioned to occur every
fiftieth year, during which slaves and prisoners would be freed, debts would be
forgiven and the mercies of God would be particularly manifest.
In Western Christianity, the tradition dates to 1300, when Pope Boniface VIII
convoked a holy year, following which ordinary jubilees have generally been
celebrated every 25 or 50 years; with extraordinary jubilees in addition
depending on need. Christian Jubilees, particularly in the Latin Church,
generally involve pilgrimage to a sacred site, normally the city of Rome. 
}
when rounding the verge of a new idea and sparkle up in happy conviction
as if to say “Excelsior!” 
\footnote{Loftier, yet higher; ever upward. From Latin comparative of
\textit{excelsus} elevated, lofty.}
He dreamed, like Astrophel, 
\footnote{Probably intended to mean "star lover", from Greek ἄστρο-ν star 
and ϕίλος lover, friend. This name was first used by the 16th-century poet 
Sir Philip Sidney in his collection of sonnets \textit{Astrophel and Stella}.
}
with his head in the stars. His mind was like one of those Gothic 
cathedrals of which he was so fond,
mysterious within, and filled with light, a brightness at once richer and less
real than the light of day, flashing accompaniment, on occasion, to the long
satirical tirades 
\footnote{A long, angry, or violent speech. A volley of words; a long and
vehement speech on some subject; a declamation; a protracted harangue, esp. of
denunciation, abuse, or invective. 
}
of which he was also capable and yet wakefully aware, in
gentleness, of what in matters of difficulty he felt should either be removed,
pitied, or understood. He was six feet tall. His hair he wore long, like the
Renaissance prince at his lyre, and it matched in color his coat of jet which
was of an obsolete but distinct cut and as black as the mundus 
\footnote{Latin for the world, universe, heavens.}
where Romans communed 
\footnote{\textit{Commune} as a verb means to converse together with sympathy
and confidence; to interchange sentiments or feelings; to take counsel.}
with their dead.

  The book he was working on --- a grimoire, 
\footnote{a manual of black magic (for invoking spirits and demons).}
in the old style --- recapitulated such
communication. He scribbled 
\footnote{To write hastily or carelessly, without regard to correctness or
elegance.}
away in the light of his gooseneck lamp that not
only left the rest of the room in darkness but at such times rendered
insignificant any matters of consequence beyond that. There was a private
quality about him as he worked: a wizard in conical 
\footnote{having the shape of a cone; Think Harry Porter.}
hat conjuring 
\footnote{\textit{Conjure} means to make (something) appear or seem to appear by
using magic.}
mastertricks; the sacristan 
\footnote{An officer of the church who has the care of the utensils or movables,
and of the church in general; a sexton.}
jing-jing-jingling the bells of sext; 
\footnote{Sext, or Sixth Hour, is a fixed time of prayer of the Divine Office of
almost all the traditional Christian liturgies. It consists mainly of psalms and
is said at noon. Its name comes from Latin and refers to the sixth hour of the
day after dawn. }
the alchemist, counsel to caliphs, 
\footnote{the civil and religious leader of a Muslim state considered to be a
representative of Allah on earth.}
shuttling in a cellar enigmatic beaker 
\footnote{a large drinking container with a large mouth.}
to tort 
\footnote{\textdbend Apparently used as a verb here, but OED does not list any
usage of tort as a verb.}
for rare demulcents 
\footnote{a medication (in the form of an oil or salve etc.) that soothes inflamed or
injured skin.}
and rubefacients. 
\footnote{a medicine for external application that produces redness of the skin.}
It was a closed world, his, arresting thoughts for words to work,
to skid 
\footnote{slide without control}
around, to transubstantiate: 
\footnote{to change into another substance.}
the writer is the ponce 
\footnote{British slang. A man employed by a prostitute to find clients, and 
who may also act as a bodyguard and driver. A ponce is different from a pimp 
in being the prostitute's employee, not the employer.
}
who introduces Can to Ought. He crafted his writing and loved listening to 
those tiny explosions when the active brutality of verbs in revolution raced 
into sweet established nouns to send marching across the page a newly 
commissioned army of words-on-maneuvers, all decorated in loops, frets, 
\footnote{n. an ornament or ornamental work often in relief consisting of small
straight bars intersecting one another in right or oblique angles.
}
and arrowlike flourishes.
\footnote{A florid decoration; a piece of scroll-work, tracery, or the like.}
Darconville was bedeviled by angels: they stalked and leaguered him by night and
day, and, when sitting at his desk, he never failed to acknowledge Stimulator,
the angel invoked in the exorcism 
\footnote{The action of exorcizing or expelling an evil spirit by adjuration or 
the performance of certain rites; an instance of this. From late L.
\textit{exorcism-us}, a. Eccl. Gr. ἐξορκισµός, f. ἐξορκίζειν.}
of ink, for the storm and stress of making
something from nothing partook 
\footnote{p.p. of \textit{partake}, To take a part, portion, lot, or share, in
common with others; to have a share or part; to participate; to share.
}
no less of the supernatural than Creation itself.
Was doubt the knot in faith’s muscle? And yet faith required to fill the desert
places of an empty page? Then this was a day in September no different from any
others on which he wrote, the mind making up madness, the hand its little
prattboy \footnote{Threoux's coinage.}
hopping along after it to record what it could of measure --- but there was
one exception.

  Darconville anxiously kept seeing the face of one of his students, someone he
had noticed on the first day of class. She was a freshman. He didn’t know her
name.

  He might have spent an inattentive afternoon in consequence but, subdued by
what had no charm for him, instead vexed 
\footnote{To make angry or annoyed by little provocations; to irritate; to
plague; to torment; to harass; to afflict; to trouble; to tease.
}
himself to write as a means of serving
notice to a mischief he’d been uncertain of now for too many days. Beauty, while
it haunted him, also distracted him; unable to resist its appeal, he, however,
longed to be above it. There is a will so strong as to recoil upon itself and
fall into indecision: a deliberate person’s, often, who, otherwise prompt to
action, sometimes leaves everything undone --- or, better, assumes that 
whatever has
been done is something already charged to an appointed end, relieving him then
of calling into question by subsequent thought the meaning of its worth. Did
Darconville’s mind, then, obsessed and overwhelmed by images and dreams of the
supernatural, crave at last for the one thing stranger than all these --- the
experience of it in fact? It is perhaps easy to believe so.

  He was born --- of French and Italian parentage --- on the reaches of coastal New
England where the old Victorian house that was the family seat stands to this
day in a small village hard by the sea. It was always a region of spectacular
beauty, infinite skies and meadows and ocean, and all aspects of nature there
seemed drawn together in a tie of inexpressible benediction. His youthful dreams
were always of a supernatural cast, shot through with vision, and nothing
whatsoever matter-of-fact could avail against the propagation of his early
romantic ideals. It had been instilled in him early --- his Venetian grandmother
fairly threw her hands over her ears at the suggestion of any aspiration less
noble --- that the goal of a person’s life must naturally afford the light by which
the rest of it should be read, a doctrine that paradoxically created in him less
a strength against than a disposition to a belief in unreal worlds, a condition
somehow making him particularly unsuited for the heartache of real life.

  The facts of one’s childhood are always important when touching on a genius.
Darconville was an ardently religious boy, much attracted to ritual. At six, he
won the school ribbon for a drawing of the face of God --- it resembled a cat’s
--- and illustrated a juvenile book of his own dramatic making which ended: 
“But wait, there is something coming toward me—!” There were illnesses, and a double
pneumonia in childhood following a nearly fatal bout with measles 
\footnote{an infectious viral disease causing fever and a red rash on the skin,
typically occurring in childhood.}
left his lungs imperfect. He would never forget his father who read to him or his mother who
kissed him goodnight, for he lost both of them before he was fourteen,
whereupon, becoming wayward 
\footnote{Taking one's own way; disobedient; froward; perverse; willful.}
and discontented with everything else, he cut short
his schooling to join the Franciscan Order, less on the advice growing out of
the newly assumed regency of his grandmother, though that, than on the
investigation of a dream: a nostalgia for vision, if commonly absent in others,
then not so for a little boy whose earliest memory was of trying to pick up
pieces of moonlight that had fallen through the window onto his bed.

  Seminary life in those early years led to strange and unaccountable
antipathies. 
\footnote{strong feeling of dislike.}
It was not that he didn’t feel he fit, but if particulars went
well --- in everything, save, perhaps, for the occasional youthful temptation he
suffered during lectio 
\footnote{Latin for reading.}
divina while reading Lucretius 
\footnote{Titus Lucretius Carus (c. 99 BC – c. 55 BC) was
a Roman poet and philosopher. His only known work is the epic philosophical poem
De rerum natura about the tenets and philosophy of Epicureanism, and which is
usually translated into English as On the Nature of Things. 
}
on the terminology of
physical love --- general acceptance came hard. He improvised 
\footnote{To bring about, arrange, do, or make, immediately or on short notice,
without previous preparation and with no known precedent as a guide.
}
piano arrangements at
midnight, claimed he could work curses, and put it about that he believed
animals, because of a universal language from which we alone had fallen, could
understand us when we spoke. He astonished his fellow students, furthermore,
with several rapturous edificial schemes few shared: to rebuild the tomb of St.
John at Ephesus; 
to set up birdhouses for Christ through upstate New York; and
to reconstruct --- he actually stepped off the dimensions on the ballfield and began
to assemble planks --- Noah’s Ark. Throughout these years he showed a splendid but
innocent sense of fun.

  Darconville owned a great fat pen he called “The Black Disaster” --- an object,
he demoted, no other hand dare touch! His classmates were solemnly,
ceremoniously, assured it was magic, and it was coveted by all of them only in
so far as stealing it might render its owner a less vivid, if not less
bumptious, 
\footnote{Obtrusively pushy; self-assertive to a pretentious extreme.}
antagonist. He managed never to relinquish 
\footnote{let go}
it, however, and drew
angels all over his copybooks; wrote squibs 
\footnote{A sarcastic speech or publication; a petty lampoon; a brief, witty
essay.}
about some of his colleagues which,
signed “Aenigmaticus,” he secretly distributed in various library books; and one
day, for drawing a fresco --- his capolavoro
\footnote{Italian for masterpiece.}
--- of St. Bernard excommunicating a
multitude of flies at Foigny, where each little creature ingeniously, but
undisguisedly, bore the face of one of the college prefects, 
\footnote{a chief officer or chief magistrate.}
he was slapped so
hard by a certain Father Theophane that it effected a stammer in him that would
be activated, during moments of confrontation, for the rest of his life.
Hostility eventually built up, and his unconventional conduct became the subject
of such unfavorable comment in the college that it was suggested he leave. A few
defended him. (He believed it was because there had once been a cardinal in his
family, as indeed there had been.) The rest, some silently abusive, naggingly
malevolent, or outright vindictive, more or less concurred in the bizarre if
hard to be seriously taken fiat 
\footnote{An authoritative command or order to do something; an effectual
decree.}
that he not only vacate the premises but
withdraw, meditate, and summarily impale himself on the same wretched object
that had been the source, in several ways, less of any black disaster than of
their own humorless and over-pious objections.

  As it happened, he never attained to the priesthood—not, however, because he
didn’t again try. For try again he did, but failed once more. And yet with what
reckless audacity, 
\footnote{Boldness, daring, intrepidity; confidence. }
with what fierce, uncompromising passion did he always charge
and fight and charge again! Resignation to appointed ends? He was not of an age
for that. And as there hovered 
\footnote{transf. and fig. To keep hanging or lingering about (a person or place), 
to wait near at hand, move to and fro near or around, as if waiting to land or alight;
also said of things intangible.
 }
before him, always, a sense of disgust in
resigning the soul to the pleasures and idle conveniences of the world, his
aspirations, individual and metaphysical, led abruptly to another decision. He
entered a Trappist 
\footnote{The Order of Cistercians of the Strict Observance (O.C.S.O.: 
Ordo Cisterciensis Strictioris Observantiae) is a Roman Catholic religious 
order of cloistered contemplative monastics who follow the Rule of 
St. Benedict. A branch of the Order of Cistercians, they have communities 
of both monks and nuns, commonly referred to as Trappists and Trappistines, 
respectively.
}
monastery at eighteen and yet, again, before long fell into
confusion and a particular variety of quarrels there for which he was never
directly responsible but to which, as we’ve seen, even the most saintly
precincts 
\footnote{esp. in pl., often applied more vaguely to the region lying 
immediately around a place, without distinct reference to any enclosure; 
the environs. 
}
are liable. The principal agent of the worst of them was a priggish
\footnote{exaggeratedly proper}
anathemette 
\footnote{\textdbend From \textit{anathema}, i.e. a detested female person.
}
named Frater Clement who, without the gift of reason, much less the
gift of faith, had as his goal not the salvation of his soul but the acquisition
in matrimony 
\footnote{the rite of marriage; the action of marrying.}
of a blond boy, another novice, who thought the world was run and
possibly owned by the Order of Cistercians. 
His submission was naïveté, but Darconville grew impatient with the other’s venal 
\footnote{Of blood: contained in the veins.}
disability. And one afternoon as the monks were proceeding to nones, 
\footnote{Eccl. A daily office, originally said at the ninth hour of the day
 (about 3 p.m.), but in later use sometimes earlier. 
 }
he snared 
\footnote{To capture (small wild animals, birds, etc.) in a snare; to catch by entangling. }
the hypocrite by the cowl,
\footnote{A garment with a hood (vestis caputiata), worn by monks, varying in 
length in different ages and according to the usages of different orders, but 
‘having the permanent characteristics of covering the head and shoulders, 
and being without sleeves’ (Cath. Dict.). †Also, formerly, a cloak or frock worn 
by laymen or by women.
}
pulled him into a side-cloister, 
\footnote{An enclosed place or space, enclosure; close;}
and --- not without stuttering --- adroitly 
\footnote{In an adroit manner; with ready skill; dexterously, cleverly.}
gave him
the lecture on spiritual discipline he found later, much to his own grief, he
himself couldn’t follow, for he saw he couldn’t forgive Frater Clement, whose
jug ears
\footnote{British slang - describes ears that stick right out to the side like the 
handles of a milk jug.}
alone at chant and chapter phosphorized 
\footnote{To combine or impregnate with phosphorus; \textit{phosphorus} means
any substance or organism that phosphoresces or shines of itself (naturally, or 
when heated, etc.); esp. (in later use) a substance that absorbs sunlight, and 
shines in the dark. 
}
his charity on the spot. Dismal
depression followed. He received the blessing of the Abbot, who repeated the
famous words of the old desert fathers: “fuge, tace, et quiesce” 
\footnote{Latin for "Flee, be silent, rest".}
and, leaving
the following day in an egg truck --- with one suitcase, a great fat pen, and all
his limitations --- went bouncing over the hill in the direction of the declining
sunshine.

  At twenty, in what was probably the climactic event of his life, Darconville
discovered writing, the sole subject of his curiosity at this time being words
and the possibility of giving expression to them. Now, among those fragile loves
to which most men look back with tenderness and passion, certain must be singled
out as of special importance: in young Darconville’s life it was to be his proud
and irrepressible grandmama whose affection for him was always on the increase
and who --- never once having failed to give fortitude to his individuality,
although in quaint 
\footnote{}
deference 
\footnote{}
to his family’s nobility on the paternal side she
used only his surname in matters of address, a habit he would continue all his
life --- with rising emphasis that gave words to his inward instincts encouraged him
at this critical juncture of his life to go live with her in Venice. At every
point she was replete 
\footnote{}
with wise suggestions, the value of which he recognized
and the tenor 
\footnote{}
of which he followed. Did he want to write? she had asked and upon
the instant answered Darconville, who had but to follow the direction of her
raised and superintendent cane to a corner where sat a beautiful desk.

  It wasn’t long before his grandmother passed away. The palazzo, immediately
becoming the object of what had even long before been a curious litigation, was
locked up. And so with a certain amount of money earmarked for his education,
belated for the clerical years, he took her cat, Spellvexit—his sole companion
now—and set sail back to the United States, looking once more for the
possibilities of the possible as possible. The spirit of his youthful dreams,
long, strangely enough, having retarded his purposes rather than advancing them,
he studiedly refused to renounce: of justice and fair play, of living instead of
dying, of loving instead of hating. Single virtue, he always believed, was proof
against manifold vice. And yet all the caprices and aspects of human life that
gratified curiosity and excited surprise in him continued only as incidents on
the way to Glory.

  Darconville—wherever—quite happily chose to live within his own world, w’thin
his own writing, within himself. The thickest, most permanent wall dividing him
from his fellow creatures was that of mediocrity. His particular sensibility
forbade him to accept unquestioned society’s rules and taboos, its situational
standards and ethics, syntheses that to him always seemed either too exclusive
or too inclusive. His domineering sense of right, as sometimes only he saw it,
and his ardent desire to keep to the fastness of his own destiny, set him apart
in several ways. Reclusive, he shrank from all avoidable company with others—it
was the prerogative of his faith to recognize, and of his character to
overpower, objection here—and chose to believe only that somewhere, perhaps on
the footing of schoolmaster, he could inoffensively foster sums, if modest, then
at least sufficient to allow him the time to write. He sought the land of
Nusquamia, a place broadly mapped out in James 4:4, and whether by chance or
perchance by intention one day, wasting no time balancing or inquiring, he
selected a school for the purpose, was hired, and disappeared again into the
arcane. It didn’t really have to have a name. In fact, however, it did. It was a
town in Virginia, called Quinsyburg.
  The train whistle there every evening seemed to beckon, dusk, precreating a
mood of sudden melancholy in a wail that left its echo behind like the passing
tribute of a sigh. And Darconville, while yet amply occupied, was by no means so
derogate from the common run of human emotions as not to share, upon hearing
it—Spellvexit always looked up—a derivative feeling of loneliness, a disposition
compounded, further, not only by the portentous evidences of the season but also
by the bleakness of the place upon which it settled. The town was the quotidian
co-efficient of limbo: there was no suddenness, no irresistibility, no velocity
of extraordinary acts. He found hours and hours of complete solitude there,
however, and that became the source, as he wrote, not of oppressive
exclusiveness but of organizing anticipations he could accommodate in his work:
the mystic’s rapture at feeling his phantom self. He had assumed this exile not
with the destitution of spirit the prodigal is too often unfairly assigned, nor
from any aristocratic weariness a previous life in foreign parts might have
induced, but rather to pull the plug of consequence from the sump of the
world—to avoid the lust of result and the vice of emulation.
  There are advantages to being in a backwater, and at the margins, in the less
symphonic underground, recriminations were few, ambition didn’t mock useful
toil, and the bald indices of failure and success became irrelevant. The man
beyond the context of hope is equally beyond the context of despair, and the
serious vow Darconville had once made to himself, medievally sworn in the old
ipsedixitist tradition of silent knights, holy knights, aimed to that still
point; so it was with love as with loneliness: to fall in love would make him a
pneumatomachian—an opponent of the spirit which, however, to him disposed to it,
nightly blew its unfathomable afflatus down the cold reaches of the otherwise
impenetrable heavens to quicken man to magic.
  It didn’t matter where he was. No, the best attitude to the world, he felt,
unless the Patristics belied us, was to look beyond it. Darconville was below
envy and above want. And what pleasures a place denied to the sight, he hoped,
were given necessarily to the imagination. He sought the broom of Eucrates, the
sword of Fragarach, the horse of Pacolet. Prosperity, furthermore, had perhaps
killed more than adversity, an observance fortified in him by what was not only
the d’Arconville motto but also his grandmother’s most often repeated if
somewhat overly enthusiastic febrifuge: “Un altro, un altro, gran’ Dio, ma più
forte!” And so he had come to this plutonial grey area, a neglected spot, where
passersby didn’t look for art to happen as it might and when it would—to lose
himself for good, in both senses, and realize the apocalypse that is
incomprehensible without Patmos. The passion for truth is unsociable. We are in
this world not to conform to it.
  It had grown dark.
  Darconville had finished a day’s writing, took some cigarettes from his
suitcase, still as yet unpacked, and walked through the disheveled light down
the flight of stairs to the porch—the night was positively beautiful—when past
the hedges, through the rustling leaves by the large tree, he thought he saw a
girl, looking apprehensively side to side, walk quickly across the street like a
tapered dream-bird in fragile but pronounced strides and then disappear. But he
noticed something else. He reached down to pick up from the doorstep a small
round object, studded with a hundred cloves, its pure odor a sweet orange like
September. It was a pomander ball. Darconville, by matchlight, slipped the
accompanying card out of its tiny envelope. It read simply: “For the fairest.”

  They were the three words that had started the Trojan War.
